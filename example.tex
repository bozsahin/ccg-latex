\documentclass[11pt]{article}
% Use of  CCG latex commands --cem bozsahin 2017
\usepackage{mathptmx} % this font is for demo. CM fonts look ugly.
\usepackage{ccg-latex}

\begin{document}
\noindent{\Large\bf Using the \verb|ccg-latex.sty| file}\hfill{\small Cem Boz\c{s}ahin}\hfill\today\medskip\bigskip

\noindent \verb|ccg-latex.sty| neither loads nor requires  packages or fonts. Kerning control for typesetting CCG slashes, LF's colon, and for rule decorations in CCG derivations is in the \verb|pt| unit, without the \verb|\!| command. This is because the eye-pleasing slash distance seems to be independent of font size, at least to my eyes.

If you feel like changing it, just look for the \verb|\kern| command. 

All math is introduced by \verb|\ensuremath| in \verb|ccg-latex.sty|; 
there is no \verb|$..$| unless you introduce it manually. I don't recommend
using math without \verb|\ensuremath| in \verb|ccg-latex.sty|.
The short form \verb|\bm{..}|, for `bare math', is for that.

To use this CCG style, please include the following line somewhere in the \LaTeX\,preamble:\medskip

\begin{verbatim}
\usepackage{ccg-latex}
\end{verbatim}\medskip

I demonstrate examples with increasingly high-level ccg-latex code, from lowest
level \verb|\cgex| to \verb|\begin{ccg}| and \verb|\begin{ccgg}|.\bigskip

An Example with the \verb|\cgex{n}{derivations}| command. The \verb|\lf{}| is already in math mode.
Lexical assumption lines are drawn by one command.
\bigskip

\cgex{3}{John & likes & Mary\\
\cglines{3}\\
\cgf{S\fs(S\bs NP)} & \cgf{(S\bs\cgs{NP}{3s})\fs NP} & \cgf{(S\bs NP)\bs((S\bs NP)\fs NP)}\\
\lf{\lambda p.p\,\so{john}} & \lf{\lambda x\lambda y.\so{like}xy} & \lf{\lambda p.p\,\so{mary}}\\
&\cgline{2}{\cgba}\\
&\cgres{2}{S\bs NP \lf{\lambda y.\so{like}\so{mary}y}}\\  % note that \cgres is by default in \cgf font
\cgline{3}{\cgfa}\\
\cgres{3}{S \lf{\so{like}\so{mary}\so{john}}}
}\medskip\bigskip

The ccg-latex code is:

\begin{verbatim}
\cgex{3}{John & likes & Mary\\
\cglines{3}\\
\cgf{S\fs(S\bs NP)} & \cgf{(S\bs\cgs{NP}{3s})\fs NP} 
  & \cgf{(S\bs NP)\bs((S\bs NP)\fs NP)}\\
\lf{\lambda p.p\,\so{john}} 
& \lf{\lambda x\lambda y.\so{like}xy} & \lf{\lambda p.p\,\so{mary}}\\
&\cgline{2}{\cgba}\\
&\cgres{2}{S\bs NP \lf{\lambda y.\so{like}\so{mary}y}}\\  
% note that \cgres is by default in \cgf font
\cgline{3}{\cgfa}\\
\cgres{3}{S \lf{\so{like}\so{mary}\so{john}}}
}
\end{verbatim}

\newpage
Same example with lower-level ccg-latex commands for lexical assumption lines, and with shorthanded 
type-raising notation using subscript and superscript. NB. they are typeset in math mode without needing \verb|$..$|.\bigskip

\cgex{3}{John & likes & Mary\\ % uses the alias \cat rather than \cgf above--same result
\cgul & \cgul & \cgul\\  % manually repeats the columns for comparison with \cglines above 
\cat{\cgss{NP}{3s}{\uparrow}} & \cat{(S\bs\cgs{NP}{3s})\fs NP} & \cat{S\bs(S\fs NP)}\\
\lf{\lambda p.p\,\so{john}} 
& \lf{\lambda x\lambda y.\so{like}xy} & \lf{\lambda p.p\,\so{mary}}\\
\cgline{2}{\cgfc}\\
\cgres{2}{S\fs NP \lf{\lambda x.\so{like}x\so{john}}}\\  % note that \cgres is by default in \cgf font
\cgline{3}{\cgfa}\\
\cgres{3}{\cat{S} \lf{\so{like}\so{mary}\so{john}}} % using \cat inside \cgres is nae problem
}\bigskip

The ccg-latex code is:

\begin{verbatim}
\cgex{3}{John & likes & Mary\\ 
% uses the alias \cat rather than \cgf above--same result
\cgul & \cgul & \cgul\\ 
 % manually repeats the columns for comparison with \cglines above 
\cat{\cgss{NP}{3s}{\uparrow}} 
& \cat{(S\bs\cgs{NP}{3s})\fs NP} & \cat{S\bs(S\fs NP)}\\
\lf{\lambda p.p\,\so{john}} 
& \lf{\lambda x\lambda y.\so{like}xy} & \lf{\lambda p.p\,\so{mary}}\\
\cgline{2}{\cgfc}\\
\cgres{2}{S\fs NP \lf{\lambda x.\so{like}x\so{john}}}\\  
% note that \cgres is by default in \cgf font
\cgline{3}{\cgfa}\\
\cgres{3}{\cat{S} \lf{\so{like}\so{mary}\so{john}}} 
% using \cat inside \cgres is nae problem
}
\end{verbatim}
\newpage

An example with double slashes (for morphology, etc.)\bigskip

\cgex{3}{dismiss& -ed\\
\cglines{2}\\
\cgf{\cgs{VP}{inf}\fs\cgs{NP}{}}
$:\lambda x\lambda y.\so{dismiss}\,x\,y$&
\cgf{((\cgs{S}{}\bs\cgs{NP}{agr})\fs NP)\bss(\cgs{VP}{inf}\fs NP)}
$:\lambda p\lambda x\lambda y.\so{past}(P\,xy)$\\
\cgline{2}{\cgba}\\
\cgres{2}{\cgf{(\cgs{S}{}\bs\cgs{NP}{agr})\fs\cgs{NP}{}}
\lf{\lambda x\lambda y.\so{past}(\so{dismiss}\,x\,y)}}
}
\bigskip

The ccg-latex code is below (using \verb|\cgf| instead of \verb|\cat|, which do the same, and native latex math for LF, which does the  same as \verb|\lf{..})|. NB. empty subscripts with no effect, if you keep changing the categories as I do).

The last \verb|\cgres| is a painful reminder that if you put inline math
in it using \verb|$..$|, you will get a warning from \LaTeX. Use \verb|\lf{..}| or \verb|\bm{..}|  instead.\medskip

\begin{verbatim}
\cgex{3}{dismiss& -ed\\
\cglines{2}\\
\cgf{\cgs{VP}{inf}\fs\cgs{NP}{}}
$:\lambda x\lambda y.\so{dismiss}\,x\,y$&
\cgf{((\cgs{S}{}\bs\cgs{NP}{agr})\fs NP)\bss(\cgs{VP}{inf}\fs NP)}
$:\lambda p\lambda x\lambda y.\so{past}(P\,xy)$\\
\cgline{2}{\cgba}\\
\cgres{2}{\cgf{(\cgs{S}{}\bs\cgs{NP}{agr})\fs\cgs{NP}{}}
\lf{\lambda x\lambda y.\so{past}(\so{dismiss}\,x\,y)}}
}
\end{verbatim}
\newpage

Here is one example with features galore, from Emmon Bach:\bigskip

{\footnotesize
Mary \cgex{6}{musn't & have & been & being & arrest & -ed\\
  \cglines{6}\\
  \cgf{(\cgs{S}{pres}\bs NP)\fs\cgs{VP}{1sg-pl}}
  & \cgf{\cgs{VP}{1sg-pl}\fs\cgs{VP}{en}}
   &\cgf{\cgs{VP}{en,ing}\fs\cgs{VP}{ing}}
   &\cgf{\cgs{VP}{pass,ing}\fs\cgs{VP}{pass}}
   &\cgf{\cgs{VP}{inf}\fds NP}
   &\cgf{\cgs{VP}{pass}\lds(\cgs{VP}{inf}\fs NP)}\\
   \cgline{2}{\cgfc}  &&&\cgline{2}{\cgba}\\
 \cgres{2}{(\cgs{S}{pres}\bs NP)\fs\cgs{VP}{en}} &&&\cgres{2}{\cgs{VP}{pass}}\\
   \cgline{3}{\cgfc}\\
 \cgres{3}{(\cgs{S}{pres}\bs NP)\fs\cgs{VP}{ing}}\\
 \cgline{4}{\cgfc}\\
 \cgres{4}{(\cgs{S}{pres}\bs NP)\fs\cgs{VP}{pass}}\\
 \cgline{6}{\cgfa}\\
 \cgres{6}{\cgs{S}{pres}\bs NP}
}}\bigskip

ccg-latex code:\bigskip
\begin{verbatim}
{\footnotesize
Mary \cgex{6}{musn't & have & been & being & arrest & -ed\\
  \cglines{6}\\
  \cgf{(\cgs{S}{pres}\bs NP)\fs\cgs{VP}{1sg-pl}}
  & \cgf{\cgs{VP}{1sg-pl}\fs\cgs{VP}{en}}
   &\cgf{\cgs{VP}{en,ing}\fs\cgs{VP}{ing}}
   &\cgf{\cgs{VP}{pass,ing}\fs\cgs{VP}{pass}}
   &\cgf{\cgs{VP}{inf}\fds NP}
   &\cgf{\cgs{VP}{pass}\lds(\cgs{VP}{inf}\fs NP)}\\
   \cgline{2}{\cgfc}  &&&\cgline{2}{\cgba}\\
 \cgres{2}{(\cgs{S}{pres}\bs NP)\fs\cgs{VP}{en}} 
 &&&\cgres{2}{\cgs{VP}{pass}}\\
   \cgline{3}{\cgfc}\\
 \cgres{3}{(\cgs{S}{pres}\bs NP)\fs\cgs{VP}{ing}}\\
 \cgline{4}{\cgfc}\\
 \cgres{4}{(\cgs{S}{pres}\bs NP)\fs\cgs{VP}{pass}}\\
 \cgline{6}{\cgfa}\\
 \cgres{6}{\cgs{S}{pres}\bs NP}
}}
\end{verbatim}

\newpage

Same example with \verb|\begin{ccg}{n}{data}{derivations}\end{ccg}|.

No gloss line, and lexical assumption lines are drawn by default.\bigskip

{\footnotesize
Mary 
\begin{ccg}{6}{musn't & have & been & being & arrest & -ed} 
  {\cgf{(\cgs{S}{pres}\bs NP)\fs\cgs{VP}{1sg-pl}}
  & \cgf{\cgs{VP}{1sg-pl}\fs\cgs{VP}{en}}
   &\cgf{\cgs{VP}{en,ing}\fs\cgs{VP}{ing}}
   &\cgf{\cgs{VP}{pass,ing}\fs\cgs{VP}{pass}}
   &\cgf{\cgs{VP}{inf}\fds NP}
   &\cgf{\cgs{VP}{pass}\lds(\cgs{VP}{inf}\fs NP)}\\
   \cgline{2}{\cgfc}  &&&\cgline{2}{\cgba}\\
 \cgres{2}{(\cgs{S}{pres}\bs NP)\fs\cgs{VP}{en}} 
 &&&\cgres{2}{\cgs{VP}{pass}}\\
   \cgline{3}{\cgfc}\\
 \cgres{3}{(\cgs{S}{pres}\bs NP)\fs\cgs{VP}{ing}}\\
 \cgline{4}{\cgfc}\\
 \cgres{4}{(\cgs{S}{pres}\bs NP)\fs\cgs{VP}{pass}}\\
 \cgline{6}{\cgfa}\\
 \cgres{6}{\cgs{S}{pres}\bs NP}
}
\end{ccg}
}\bigskip

ccg-latex code:\bigskip

\begin{verbatim}
{\footnotesize
Mary 
\begin{ccg}{6}{musn't & have & been & being & arrest & -ed} 
  {
  \cgf{(\cgs{S}{pres}\bs NP)\fs\cgs{VP}{1sg-pl}}
  & \cgf{\cgs{VP}{1sg-pl}\fs\cgs{VP}{en}}
   &\cgf{\cgs{VP}{en,ing}\fs\cgs{VP}{ing}}
   &\cgf{\cgs{VP}{pass,ing}\fs\cgs{VP}{pass}}
   &\cgf{\cgs{VP}{inf}\fds NP}
   &\cgf{\cgs{VP}{pass}\lds(\cgs{VP}{inf}\fs NP)}\\
   \cgline{2}{\cgfc}  &&&\cgline{2}{\cgba}\\
 \cgres{2}{(\cgs{S}{pres}\bs NP)\fs\cgs{VP}{en}} 
 &&&\cgres{2}{\cgs{VP}{pass}}\\
   \cgline{3}{\cgfc}\\
 \cgres{3}{(\cgs{S}{pres}\bs NP)\fs\cgs{VP}{ing}}\\
 \cgline{4}{\cgfc}\\
 \cgres{4}{(\cgs{S}{pres}\bs NP)\fs\cgs{VP}{pass}}\\
 \cgline{6}{\cgfa}\\
 \cgres{6}{\cgs{S}{pres}\bs NP}
}
\end{ccg}
}
\end{verbatim}
\newpage


\noindent Another example, to show glossing in the beginning. The end gloss is typeset by \verb|\mc|, for multi-column, centered. 

It uses
\verb|\begin{ccgg}{n}{data}{gloss}{derivations}\end{ccgg}|.
\bigskip

\begin{ccgg}{3}{ver-dir & -t & -ti.}{give{-caus} & {-caus} & {-past}}
{
\cgf{\cgs{VP}{inf}\bs\cgs{NP}{dat}\bs\cgs{NP}{dat}\bs\cgs{NP}{acc}}
& \cgf{(S\bs\cgs{NP}{nom}\bs\cgs{NP}{case})\bs\cgs{VP}{inf}}\\
\lf{\lambda x\lambda y\lambda z.\so{give}yxz} & \lf{\lambda p\lambda x\lambda y.\so{cause}(px)y}\\
\cgline{2}{\cgbc$^3$}\\ \cgres{2}{S\bs\cgs{NP}{nom}\bs\cgs{NP}{dat}\bs\cgs{NP}{dat}\bs\cgs{NP}{dat}\bs\cgs{NP}{acc}}\\
\cgres{2}{\lf{\lambda x_1\lambda x_2\lambda x_3\lambda x_4\lambda x_5.\so{cause_{1,2}}(\sos{cause}{1,2}(\so{give}x_1x_2x_3)x_4)x_5}}\\[1ex]
\mc{3}{`made to let give', from Turkish}
}
\end{ccgg}\bigskip

ccg-latex code:\bigskip

\begin{verbatim}
\begin{ccgg}{3}{ver-dir & -t & -ti.}{give{-caus} & {-caus} & {-past}}
{
\cgf{\cgs{VP}{inf}\bs\cgs{NP}{dat}\bs\cgs{NP}{dat}\bs\cgs{NP}{acc}}
& \cgf{(S\bs\cgs{NP}{nom}\bs\cgs{NP}{case})\bs\cgs{VP}{inf}}\\
\lf{\lambda x\lambda y\lambda z.\so{give}yxz} 
& \lf{\lambda p\lambda x\lambda y.\so{cause}(px)y}\\
\cgline{2}{\cgbc$^3$}\\ 
\cgres{2}{S\bs\cgs{NP}{nom}\bs
  \cgs{NP}{dat}\bs\cgs{NP}{dat}\bs\cgs{NP}{dat}\bs\cgs{NP}{acc}}\\
\cgres{2}{\lf{\lambda x_1\lambda x_2\lambda x_3\lambda x_4\lambda x_5.
\so{cause_{1,2}}(\sos{cause}{1,2}(\so{give}x_1x_2x_3)x_4)x_5}}\\[1ex]
\mc{3}{`made to let give', from Turkish}
}
\end{ccgg}
\end{verbatim}\bigskip

Note that subscripted semantic objects such as \so{cause_{1,2}} are better typeset
as \sos{cause}{1,2} using \verb|\sos| (for `semantic object, subscripted'), rather than \verb|\so|.
\newpage

An example with slash modalities and centered non-derivability indicator (*):\bigskip

\begin{ccg}{4}{*I~will& give& flowers& my~very~heavy~friends.}
{
&\cat{(VP\fs NP)\fds NP}&\cat{VP\bs(VP\fs NP)}&\cat{VP\bs(VP\fs NP)}\\
&\badline{2}{\cgbx}
}
\end{ccg}\bigskip

Code:\bigskip

\begin{verbatim}
\begin{ccg}{4}{*I~will& give& flowers& my~very~heavy~friends.}
{
&\cat{(VP\fs NP)\fds NP}&\cat{VP\bs(VP\fs NP)}&\cat{VP\bs(VP\fs NP)}\\
&\badline{2}{\cgbx}
}
\end{ccg}
\end{verbatim}
\newpage

Some examples to show slash distancing by \verb|pt| unit for categories in subscript:\bigskip

John [should]\bm{_{\mbox{\scriptsize\cat{(S\bs NP)\rds IV}}}}  walk the dog.

John [wants]\bm{_{\mbox{\scriptsize\cat{(S\lds NP)\fs VP}}}} to walk the dog.

Mary [believes]\bm{_{\mbox{\scriptsize\cat{(S\bs NP)\fds S}}}} that John walked the dog.\bigskip

And some big ones:\bigskip

{\large
\cat{(S\bxs NP)\fds IV}\smallskip

\cat{(S\lds NP)\fs VP}\smallskip

\cat{(S\bs NP)\rds S}\smallskip
}\bigskip

and bigger ones:\bigskip

{\Large
\cat{(S\bxs NP)\fds IV}\smallskip

\cat{(S\lds NP)\fs VP}\smallskip

\cat{(S\bs NP)\rds S}
}\bigskip

Here is the native \LaTeX\, math rendering of CCG categories without \verb|ccg-latex| to compare spacing in various sizes:\bigskip

{\Large 
$(S\backslash_\times NP)/_{\diamond}IV$\medskip

$(S\backslash\backslash NP)/VP$\medskip

$(S\backslash NP)// S$}\bigskip


John [should]$_{\scriptstyle (S\backslash NP)//IV}$  walk the dog.

John [wants]$_{\scriptstyle (S\backslash\backslash NP)/ VP}$ to walk the dog.

Mary [believes]$_{\scriptstyle (S\backslash NP)/_{\diamond}S}$ that John walked the dog.
\newpage

Check the code:\footnotesize

\begin{verbatim}
{\Large 
$(S\backslash_\times NP)/_{\diamond}IV$\medskip

$(S\backslash\backslash NP)/VP$\medskip

$(S\backslash NP)// S$}\bigskip


John [should]$_{\scriptstyle (S\backslash NP)//IV}$  walk the dog.

John [wants]$_{\scriptstyle (S\backslash\backslash NP)/ VP}$ to walk the dog.

Mary [believes]$_{\scriptstyle (S\backslash NP)/_{\diamond}S}$ that John walked the dog.
\end{verbatim}\bigskip


\normalsize It generates. \bigskip

{\Large 
$(S\backslash_\times NP)/_{\diamond}IV$\medskip

$(S\backslash\backslash NP)/VP$\medskip

$(S\backslash NP)// S$}\bigskip


John [should]$_{\scriptstyle (S\backslash NP)//IV}$  walk the dog.

John [wants]$_{\scriptstyle (S\backslash\backslash NP)/ VP}$ to walk the dog.

Mary [believes]$_{\scriptstyle (S\backslash NP)/_{\diamond}S}$ that John walked the dog.\bigskip\bigskip

Not pretty.


\end{document}
